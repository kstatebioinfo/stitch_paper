%% BioMed_Central_Tex_Template_v1.06
%%                                      %
%  bmc_article.tex            ver: 1.06 %
%                                       %

%%IMPORTANT: do not delete the first line of this template
%%It must be present to enable the BMC Submission system to
%%recognise this template!!

%%%%%%%%%%%%%%%%%%%%%%%%%%%%%%%%%%%%%%%%%
%%                                     %%
%%  LaTeX template for BioMed Central  %%
%%     journal article submissions     %%
%%                                     %%
%%          <8 June 2012>              %%
%%                                     %%
%%                                     %%
%%%%%%%%%%%%%%%%%%%%%%%%%%%%%%%%%%%%%%%%%


%%%%%%%%%%%%%%%%%%%%%%%%%%%%%%%%%%%%%%%%%%%%%%%%%%%%%%%%%%%%%%%%%%%%%
%%                                                                 %%
%% For instructions on how to fill out this Tex template           %%
%% document please refer to Readme.html and the instructions for   %%
%% authors page on the biomed central website                      %%
%% http://www.biomedcentral.com/info/authors/                      %%
%%                                                                 %%
%% Please do not use \input{...} to include other tex files.       %%
%% Submit your LaTeX manuscript as one .tex document.              %%
%%                                                                 %%
%% All additional figures and files should be attached             %%
%% separately and not embedded in the \TeX\ document itself.       %%
%%                                                                 %%
%% BioMed Central currently use the MikTex distribution of         %%
%% TeX for Windows) of TeX and LaTeX.  This is available from      %%
%% http://www.miktex.org                                           %%
%%                                                                 %%
%%%%%%%%%%%%%%%%%%%%%%%%%%%%%%%%%%%%%%%%%%%%%%%%%%%%%%%%%%%%%%%%%%%%%

%%% additional documentclass options:
%  [doublespacing]
%  [linenumbers]   - put the line numbers on margins

%%% loading packages, author definitions

%\documentclass[twocolumn]{bmcart}% uncomment this for twocolumn layout and comment line below
\documentclass{bmcart}

%%% Load packages
%\usepackage{amsthm,amsmath}
%\RequirePackage{natbib}
%\RequirePackage{hyperref}
\usepackage[utf8]{inputenc} %unicode support
%\usepackage[applemac]{inputenc} %applemac support if unicode package fails
%\usepackage[latin1]{inputenc} %UNIX support if unicode package fails


%%%%%%%%%%%%%%%%%%%%%%%%%%%%%%%%%%%%%%%%%%%%%%%%%
%%                                             %%
%%  If you wish to display your graphics for   %%
%%  your own use using includegraphic or       %%
%%  includegraphics, then comment out the      %%
%%  following two lines of code.               %%
%%  NB: These line *must* be included when     %%
%%  submitting to BMC.                         %%
%%  All figure files must be submitted as      %%
%%  separate graphics through the BMC          %%
%%  submission process, not included in the    %%
%%  submitted article.                         %%
%%                                             %%
%%%%%%%%%%%%%%%%%%%%%%%%%%%%%%%%%%%%%%%%%%%%%%%%%


\def\includegraphic{}
\def\includegraphics{}



%%% Put your definitions there:
\startlocaldefs
\endlocaldefs


%%% Begin ...
\begin{document}

%%% Start of article front matter
\begin{frontmatter}

\begin{fmbox}
\dochead{Research}

%%%%%%%%%%%%%%%%%%%%%%%%%%%%%%%%%%%%%%%%%%%%%%
%%                                          %%
%% Enter the title of your article here     %%
%%                                          %%
%%%%%%%%%%%%%%%%%%%%%%%%%%%%%%%%%%%%%%%%%%%%%%

\title{A sample article title}

%%%%%%%%%%%%%%%%%%%%%%%%%%%%%%%%%%%%%%%%%%%%%%
%%                                          %%
%% Enter the authors here                   %%
%%                                          %%
%% Specify information, if available,       %%
%% in the form:                             %%
%%   <key>={<id1>,<id2>}                    %%
%%   <key>=                                 %%
%% Comment or delete the keys which are     %%
%% not used. Repeat \author command as much %%
%% as required.                             %%
%%                                          %%
%%%%%%%%%%%%%%%%%%%%%%%%%%%%%%%%%%%%%%%%%%%%%%

\author[
   addressref={aff1},                   % id's of addresses, e.g. {aff1,aff2}
   corref={aff1},                       % id of corresponding address, if any
   noteref={n1},                        % id's of article notes, if any
   email={jane.e.doe@cambridge.co.uk}   % email address
]{\inits{JE}\fnm{Jane E} \snm{Doe}}
\author[
   addressref={aff1,aff2},
   email={john.RS.Smith@cambridge.co.uk}
]{\inits{JRS}\fnm{John RS} \snm{Smith}}

%%%%%%%%%%%%%%%%%%%%%%%%%%%%%%%%%%%%%%%%%%%%%%
%%                                          %%
%% Enter the authors' addresses here        %%
%%                                          %%
%% Repeat \address commands as much as      %%
%% required.                                %%
%%                                          %%
%%%%%%%%%%%%%%%%%%%%%%%%%%%%%%%%%%%%%%%%%%%%%%

\address[id=aff1]{%                           % unique id
  \orgname{Department of Zoology, Cambridge}, % university, etc
  \street{Waterloo Road},                     %
  %\postcode{}                                % post or zip code
  \city{London},                              % city
  \cny{UK}                                    % country
}
\address[id=aff2]{%
  \orgname{Marine Ecology Department, Institute of Marine Sciences Kiel},
  \street{D\"{u}sternbrooker Weg 20},
  \postcode{24105}
  \city{Kiel},
  \cny{Germany}
}

%%%%%%%%%%%%%%%%%%%%%%%%%%%%%%%%%%%%%%%%%%%%%%
%%                                          %%
%% Enter short notes here                   %%
%%                                          %%
%% Short notes will be after addresses      %%
%% on first page.                           %%
%%                                          %%
%%%%%%%%%%%%%%%%%%%%%%%%%%%%%%%%%%%%%%%%%%%%%%

\begin{artnotes}
%\note{Sample of title note}     % note to the article
\note[id=n1]{Equal contributor} % note, connected to author
\end{artnotes}

\end{fmbox}% comment this for two column layout

%%%%%%%%%%%%%%%%%%%%%%%%%%%%%%%%%%%%%%%%%%%%%%
%%                                          %%
%% The Abstract begins here                 %%
%%                                          %%
%% Please refer to the Instructions for     %%
%% authors on http://www.biomedcentral.com  %%
%% and include the section headings         %%
%% accordingly for your article type.       %%
%%                                          %%
%%%%%%%%%%%%%%%%%%%%%%%%%%%%%%%%%%%%%%%%%%%%%%

\begin{abstractbox}

\begin{abstract} % abstract
\parttitle{First part title} %if any
Text for this section.

\parttitle{Second part title} %if any
Text for this section.
\end{abstract}

%%%%%%%%%%%%%%%%%%%%%%%%%%%%%%%%%%%%%%%%%%%%%%
%%                                          %%
%% The keywords begin here                  %%
%%                                          %%
%% Put each keyword in separate \kwd{}.     %%
%%                                          %%
%%%%%%%%%%%%%%%%%%%%%%%%%%%%%%%%%%%%%%%%%%%%%%

\begin{keyword}
\kwd{sample}
\kwd{article}
\kwd{author}
\end{keyword}

% MSC classifications codes, if any
%\begin{keyword}[class=AMS]
%\kwd[Primary ]{}
%\kwd{}
%\kwd[; secondary ]{}
%\end{keyword}

\end{abstractbox}
%
%\end{fmbox}% uncomment this for twcolumn layout

\end{frontmatter}

%%%%%%%%%%%%%%%%%%%%%%%%%%%%%%%%%%%%%%%%%%%%%%
%%                                          %%
%% The Main Body begins here                %%
%%                                          %%
%% Please refer to the instructions for     %%
%% authors on:                              %%
%% http://www.biomedcentral.com/info/authors%%
%% and include the section headings         %%
%% accordingly for your article type.       %%
%%                                          %%
%% See the Results and Discussion section   %%
%% for details on how to create sub-sections%%
%%                                          %%
%% use \cite{...} to cite references        %%
%%  \cite{koon} and                         %%
%%  \cite{oreg,khar,zvai,xjon,schn,pond}    %%
%%  \nocite{smith,marg,hunn,advi,koha,mouse}%%
%%                                          %%
%%%%%%%%%%%%%%%%%%%%%%%%%%%%%%%%%%%%%%%%%%%%%%

%%%%%%%%%%%%%%%%%%%%%%%%% start of article main body
% <put your article body there>

%%%%%%%%%%%%%%%%
%% Background %%
%%
\section*{Content}
Text and results for this section, as per the individual journal's instructions for authors. %\cite{koon,oreg,khar,zvai,xjon,schn,pond,smith,marg,hunn,advi,koha,mouse}

\section*{Section title}
Text for this section \ldots
\subsection*{Sub-heading for section}
Text for this sub-heading \ldots
\subsubsection*{Sub-sub heading for section}
Text for this sub-sub-heading \ldots
\paragraph*{Sub-sub-sub heading for section}
Text for this sub-sub-sub-heading \ldots
In this section we examine the growth rate of the mean of $Z_0$, $Z_1$ and $Z_2$. In
addition, we examine a common modeling assumption and note the
importance of considering the tails of the extinction time $T_x$ in
studies of escape dynamics.
We will first consider the expected resistant population at $vT_x$ for
some $v>0$, (and temporarily assume $\alpha=0$)
%
\[
 E \bigl[Z_1(vT_x) \bigr]= E
\biggl[\mu T_x\int_0^{v\wedge
1}Z_0(uT_x)
\exp \bigl(\lambda_1T_x(v-u) \bigr)\,du \biggr].
\]
%
If we assume that sensitive cells follow a deterministic decay
$Z_0(t)=xe^{\lambda_0 t}$ and approximate their extinction time as
$T_x\approx-\frac{1}{\lambda_0}\log x$, then we can heuristically
estimate the expected value as
%
\begin{eqnarray}\label{eqexpmuts}
E\bigl[Z_1(vT_x)\bigr] &=& \frac{\mu}{r}\log x
\int_0^{v\wedge1}x^{1-u}x^{({\lambda_1}/{r})(v-u)}\,du
\nonumber\\
&=& \frac{\mu}{r}x^{1-{\lambda_1}/{\lambda_0}v}\log x\int_0^{v\wedge
1}x^{-u(1+{\lambda_1}/{r})}\,du
\nonumber\\
&=& \frac{\mu}{\lambda_1-\lambda_0}x^{1+{\lambda_1}/{r}v} \biggl(1-\exp \biggl[-(v\wedge1) \biggl(1+
\frac{\lambda_1}{r}\biggr)\log x \biggr] \biggr).
\end{eqnarray}
%
Thus we observe that this expected value is finite for all $v>0$ (also see \cite{koon,khar,zvai,xjon,marg}).



%%%%%%%%%%%%%%%%%%%%%%%%%%%%%%%%%%%%%%%%%%%%%%
%%                                          %%
%% Backmatter begins here                   %%
%%                                          %%
%%%%%%%%%%%%%%%%%%%%%%%%%%%%%%%%%%%%%%%%%%%%%%

\begin{backmatter}

\section*{Competing interests}
  The authors declare that they have no competing interests.

\section*{Author's contributions}
    Text for this section \ldots

\section*{Acknowledgements}
  Text for this section \ldots
%%%%%%%%%%%%%%%%%%%%%%%%%%%%%%%%%%%%%%%%%%%%%%%%%%%%%%%%%%%%%
%%                  The Bibliography                       %%
%%                                                         %%
%%  Bmc_mathpys.bst  will be used to                       %%
%%  create a .BBL file for submission.                     %%
%%  After submission of the .TEX file,                     %%
%%  you will be prompted to submit your .BBL file.         %%
%%                                                         %%
%%                                                         %%
%%  Note that the displayed Bibliography will not          %%
%%  necessarily be rendered by Latex exactly as specified  %%
%%  in the online Instructions for Authors.                %%
%%                                                         %%
%%%%%%%%%%%%%%%%%%%%%%%%%%%%%%%%%%%%%%%%%%%%%%%%%%%%%%%%%%%%%

% if your bibliography is in bibtex format, use those commands:
\bibliographystyle{bmc-mathphys} % Style BST file
\bibliography{bmc_article}      % Bibliography file (usually '*.bib' )

% or include bibliography directly:
% \begin{thebibliography}
% \bibitem{b1}
% \end{thebibliography}

%%%%%%%%%%%%%%%%%%%%%%%%%%%%%%%%%%%
%%                               %%
%% Figures                       %%
%%                               %%
%% NB: this is for captions and  %%
%% Titles. All graphics must be  %%
%% submitted separately and NOT  %%
%% included in the Tex document  %%
%%                               %%
%%%%%%%%%%%%%%%%%%%%%%%%%%%%%%%%%%%

%%
%% Do not use \listoffigures as most will included as separate files

\section*{Figures}
  \begin{figure}[h!]
  	\caption{\csentence{Assembly workflow for assemble\_SGE\_cluster.pl.}
  		(A) The Irys instrument produces tiff files that are converted into BNX text files. (B) Each chip produces one BNX file for each of two flowcells. (C) BNX files are split by scan and aligned to the sequence reference. Stretch (bases per pixel) is recalculated from the alignment. (D) Quality check graphs are created for each pre-adjusted flowcell BNX. (E) Adjusted flowcell BNXs are merged. (F) The first assemblies are run with a variety of p-value thresholds. (G) The best of the first assemblies (red oval) is chosen and a version of this assembly is produced with a variety of minimum molecule length filters.}
  \end{figure}
  \begin{figure}[h!]
  \caption{\csentence{Steps of the stitch.pl algorithm.}
      BNG CMAPs (blue) are shown aligned to \textit{in silico} CMAPs (green). Alignments are indicated with grey lines. CMAP orientation for \textit{in silico} CMAPs is indicated with a "+" or "-" for positive or negative orientation respectively. (A) The \textit{in silico} CMAP is aligned as the reference. (B) The alignment is inverted and used as input for stitch.pl. (C) The alignments are filtered based on alignment length (purple) relative to total possible alignment length (black) and confidence. Here assuming all alignments have a high confidence score and the minimum percent aligned is 30\% two alignments fail for aligning over less than 30\% of the potential alignment length for that alignment. (D) Filtering produces an XMAP of high quality alignments with short (local) alignments removed. (E) High quality scaffolding alignments are filtered for longest and highest confidence alignment for each \textit{in silico} CMAP. Third alignment (unshaded) is filtered because the second alignment is the longest alignment for {in silico} CMAP 2. (F) Passing alignments are used to super scaffold (captured gaps indicated in dark green). (G) Stitch is iterated and additional super scaffolding alignments are found using second best scaffolding alignments. (H) Iteration takes advantage of cases where \textit{in silico} CMAPs scaffold BNG CMAPs as \textit{in silico} CMAP 2 does. Stitch is run iteratively until super scaffolding alignments are found.}
      \end{figure}      
\begin{figure}[h!]
	\caption{\csentence{Putative haplotypes assembled as BNG CMAPs.}
		(A) Two BNG CMAPs (blue with molecule coverage shown in dark blue) align to the \textit{in silico} CMAP of scaffold 131 (green with contigs overlaid as translucent colored squares). (B and C) Both BNG CMAPs are shown (blue) with molecule pileups (yellow). Both BNG CMAPs have similar label patterns except within the lower coverage region indicated with a black square.}
\end{figure}  
\begin{figure}[h!]
	\caption{\csentence{Histogram of gap lengths in Tcas5.1.}
		Positive and negative gaps lengths for Tcas5.1 added to the automated output of stitch.pl based on filtered scaffolding alignments. The majority of gap lengths added by stitch.pl, 66, were positive (red). The remaining 26 gaps had negative lengths (purple).}
      \end{figure}                      
\begin{figure}[h!]
	\caption{\csentence{Extremely small negative gap length for \textit{in silico} CMAP of scaffold 81.}
		Two XMAP alignments for \textit{in silico} CMAP of scaffold 81 are shown. BNG CMAPs (blue with molecule coverage shown in dark blue) align to the \textit{in silico} CMAPs of scaffolds (green with contigs overlaid as translucent colored squares). Scaffolds 79-83 were placed within ChLG 5 and scaffolds 99-103 were placed with ChLG 7 by the \textit{Tribolium} genetic map. (A) Half of the \textit{in silico} CMAP of scaffold 81 aligns with its assigned ChLG (black arrow). (B) The other half aligns with ChLG 7 (red arrow) producing a negative gap length smaller than -20 (kb). The alignment that places scaffold 81 with ChLG 7 disagrees with the genetic map and was manually rejected for Tcas5.2.}
      \end{figure}            


%%%%%%%%%%%%%%%%%%%%%%%%%%%%%%%%%%%
%%                               %%
%% Tables                        %%
%%                               %%
%%%%%%%%%%%%%%%%%%%%%%%%%%%%%%%%%%%

%% Use of \listoftables is discouraged.
%%

\section*{Tables}
\begin{table}[h!]
	\caption{Assembly Results. Assembly metrics for Tcas5.0 (the starting scaffolded FASTA), the \textit{in silico} CMAP, the BNG CMAP of assembled molecules and the final super scaffolded FASTA (Tcas5.2) produced using stitch.pl for the \textit{Tribolium} genome.}
	\begin{tabular}{cccc}
		\hline
		& N50 (Mb)  & Number & Cumulative Length (Mb)\\ \hline
		Genome FASTA & 1.16 & 2240 & 160.74\\
		\textit{in silico} CMAP & 1.20 &  223 & 152.53\\
		BNG CMAP & 1.35 &  216 & 200.47\\
		Super scaffold FASTA & 4.46 & 2150 & 165.92\\ \hline
	\end{tabular}
\end{table}
\begin{table}[h!]
\caption{Alignment of BNG assembly to reference genome. Breadth of alignment coverage (non-redundant alignment), length of total alignment (including redundant alignments) and percent of CMAP covered (non-redundantly) were calculated for the \textit{in silico} CMAP and the BNG CMAP of the \textit{Tribolium} genome the using xmap\_stats.pl.}
      \begin{tabular}{cccc}
        \hline
           & Breadth of alignment coverage (Mb) & Length of total alignment (Mb) & Percent of CMAP aligned \\ \hline
        \textit{in silico} CMAP from FASTA & 124.04 & 132.40 & 81 \\
        BNG CMAP & 131.64 & 132.34 & 67 \\ \hline
      \end{tabular}
\end{table}
\begin{table}[h!]
	\caption{Chromosome linkage groups before and after super scaffolding. The number of scaffolds and contigs in the Tcas5.0 ChLG bins and the number of super scaffolds, scaffolds and contigs in the Tcas5.2 ChLG bins. The number of scaffolds or contigs that were unplaced in Tcas5.0 and placed with a ChLG in Tcas5.2 is also listed. }
	\begin{tabular}{cccc}
		\hline
		Chromosome linkage group (ChLG) & Tcas5.0 scaffolds  & Tcas5.2 super scaffolds & Unplaced scaffolds added in Tcas5.2\\ \hline
		X & 13 & 2 & 2\\
		2 & 18 & 10 & 1\\
		3 & 29 & 20 & 4\\
		4 & 6 & 2 & 2\\
		5 & 17 & 4 & 1\\
		6 & 12 & 6 & 6\\
		7 & 15 & 6 & 0\\
		8 & 14 & 8 & 1\\
		9 & 21 & 9 & 0\\
		10 & 12 & 11 & 2\\ \hline
		Total & 157 & 78 & 19\\ \hline
	\end{tabular}
\end{table}



%%%%%%%%%%%%%%%%%%%%%%%%%%%%%%%%%%%
%%                               %%
%% Additional Files              %%
%%                               %%
%%%%%%%%%%%%%%%%%%%%%%%%%%%%%%%%%%%

\section*{Additional Files}
  \subsection*{Additional file 1 --- Sample additional file title}
    Additional file descriptions text (including details of how to
    view the file, if it is in a non-standard format or the file extension).  This might
    refer to a multi-page table or a figure.

  \subsection*{Additional file 2 --- Sample additional file title}
    Additional file descriptions text.


\end{backmatter}
\end{document}
